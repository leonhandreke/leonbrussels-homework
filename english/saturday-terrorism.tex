\documentclass[11pt]{article}
\usepackage{geometry}                % See geometry.pdf to learn the layout options. There are lots.
\geometry{letterpaper}
\usepackage{setspace}
%\geometry{landscape}                % Activate for for rotated page geometry
\usepackage[parfill]{parskip}    % Activate to begin paragraphs with an empty line rather than an indent
\usepackage{graphicx}

\usepackage{helvet}
\usepackage{nopageno}

%\usepackage[german]{babel}
\usepackage[T1]{fontenc}
\usepackage[utf8]{inputenc}


\DeclareGraphicsRule{.tif}{png}{.png}{`convert #1 `dirname #1`/`basename #1 .tif`.png}

\title{Saturday - In Class Essay}
\author{Leon Handreke}
\date{}                                           % Activate to display a given date or no date

\begin{document}

%double line spacing
\onehalfspacing

\maketitle
\fontfamily{phv}\selectfont

In the novel ``Saturday'' by Ian McEwan terrorism is a major theme. However, the ``private terror'' that the Perowne family experiences creates a feeling of a more direct threat to each individual. In this essay I aim to explain the effect that this shift from a remote, indirect threat such as public terrorism to a more direct, personal threat has on the characters in the book as well as the reader.

Public terrorism is without doubt a major theme in the book and something that Henry Perowne thinks about a lot over the course of the novel. Early in the morning, as he observes an airplane with a burning engine on approach to Heathrow, he immediately thinks of a terrorist attack rather than a technical malfunction of the aircraft. He also briefly touches on the issue in the discussion about the Iraq War that he has with his daughter.

However, the real threat in the book seems to come from Baxter, a mentally-ill young man seeking revenge because his wing mirror got damaged by Perowne's car. He enters Henry's house together with his two friends carrying simple weapons such as knives and threatens several members of the family, even injuring Grammaticus by breaking his nose.

The events in the novel suggest that in fact, a ``private terrorist attack'' is far more likely than a public one. The conclusion from these two events however, is left to the reader. Over the course of the novel, Henry never connects these two events. Ian McEwan may have written the novel with this exact intention - letting the reader think about the events in the book and assess the real threat for himself.

One aspect that unites both of these events and is also described by Henry is the feeling of vulnerability and the inability to help. When he sees the plane, Henry realizes that there is nothing he can do to influence the outcome (the metaphor of Schroedinger'ss cat is used to explain this feeling) and that London is very vulnerably for a terrorist bomb attack (``just waiting for it's bomb''). A similar feeling is described while Baxter is threatening his family: He is constantly thinking about ways to get out of his current situation, however realizes in the end that Baxter and his friends are stronger. Only by chance does he escape Baxter in the end.

The effect on the novel is very clear: It makes the reader realize how vulnerable we are, both as a society and as individuals.

Perhaps the most important effect that the ``private terror'' described in the novel has on the story is that it makes the threat of violent attacks to Henry and his family a lot more real. This aspect of the story might also be designed to scare the reader and cause him to think about how vulnerable and threatened he himself is.

In conclusion, I think that the contrast between these two events is an important aspect of the novel ``Saturday'' By letting the violence and feelings of the aggressor come a lot closer to Henry and the reader, Ian McEwan forms an extreme contrast between the burning airplane silently gliding far away from Henry and the reader. By letting the reader draw some conclusions from the events described by himself rather than suggesting a conclusion, he encourages the reader to further think about the novel and gives room for deeper interpretation of the story and its characters.

\end{document}


% LocalWords:  McEwan
