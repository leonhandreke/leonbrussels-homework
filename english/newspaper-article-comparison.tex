\documentclass[10pt]{article}
\usepackage{geometry}                % See geometry.pdf to learn the layout options. There are lots.
\geometry{a4paper}                   % ... or a4paper or a5paper or ... 
\usepackage{setspace}
%\geometry{landscape}                % Activate for for rotated page geometry
\usepackage[parfill]{parskip}    % Activate to begin paragraphs with an empty line rather than an indent
\usepackage{graphicx}

\usepackage{helvet}
\usepackage{nopageno}

%\usepackage[german]{babel}
\usepackage[T1]{fontenc}
\usepackage[utf8]{inputenc}


\DeclareGraphicsRule{.tif}{png}{.png}{`convert #1 `dirname #1`/`basename #1 .tif`.png}

\title{Newspaper Research - Comparing Articles}
\author{Leon Handreke}
\date{}                                           % Activate to display a given date or no date

\begin{document}

%double line spacing
\doublespacing

\maketitle
\fontfamily{phv}\selectfont

Although the two newspaper article in question report about the same topic and take up about the same space on the page, there are striking differences between them.

One difference that becomes obvious at first sight is the headline; The first article uses a big, bold font for the heading which only consists of 6 words. The word ``Chaos'' is probably used to attract the reader and motivate him to read further. The second article tries to capture the reader with a clever play of words and features a smaller font which leaves more space for the actual article.

Another difference is the style of writing. The first article uses a lot of informal language and reports about facts that are often not very important for the reader but are interesting. An example would be the hurt fireman who was ``knocked down by a freak wave''. The second article gives a lot more information about the subject, what happened and where. It uses more formal language that is still easy to read.

The first newspaper article is more aimed at the people who want easy to read, short and entertaining news. It gives just the right amount of information to the average reader who is not very interested and wants to be entertained. In contrast, the second article contains a lot of information, probably more than a reader who ist not particularily interested wants to read. However, it gives plenty of well-researched, in-depth information.

From looking at the articles, it can be concluded that the first one is more likely to be printed in a tabloid while the second one containing more information but is less entertaining is probably taken from a broadsheet newspaper. 

\end{document} 

