\documentclass[10pt]{article}
\usepackage{geometry}                % See geometry.pdf to learn the layout options. There are lots.
\geometry{a4paper}                   % ... or a4paper or a5paper or ...
\usepackage{setspace}
%\geometry{landscape}                % Activate for for rotated page geometry
\usepackage[parfill]{parskip}    % Activate to begin paragraphs with an empty line rather than an indent
\usepackage{graphicx}

\usepackage{helvet}
\usepackage{nopageno}

%\usepackage[german]{babel}
\usepackage[T1]{fontenc}
\usepackage[utf8]{inputenc}

\DeclareGraphicsRule{.tif}{png}{.png}{`convert #1 `dirname #1`/`basename #1 .tif`.png}

\title{Report Writing - European School}
\author{Leon Handreke}
\date{}                                           % Activate to display a given date or no date

\begin{document}

\onehalfspacing

\maketitle
\fontfamily{phv}\selectfont

\textbf{Report to:} European School's Board of Directors

\textbf{Concerning:} European School Brussels II (Woluwe)

\textbf{Findings:} Despite the relatively old age of the school, the buildings are still in relatively good condition. New investments are being made in the field of representative buildings and green energy. Some classrooms and part of the sport facilities however are in relatively bad condition.

Lunching facilities for students have improved since 2007. A bar to buy sandwiches has been introduced, however is still facing some problems such as that almost all sandwiches have been sold by mid day. Also, the name ``Cafeteria'' implies the availability of caffeine products, which is not the case at this time. The APEE-run canteen quality has not improved significantly but has increased the price per lunch again.

Tutition quality has stayed constant and is still noticably worse in the first years of secondary school. Surveys among pupils have indicated that satisfaction levels are significantly higher in years 5, 6 and 7, indicating an uneven spread of qualified teachers.

The school has also continued to encourage a ``green lifestyle'', installing solar pannels on its roof, further improving the bus system and adjusting the curriculum to increase awareness for environmental problems. However, facilities for students coming to school by bike are still underdeveloped and exposed to vandalism by primary pupils.

\textbf{Conclusion and Recommendations:} More money should be spent on improving the tuition facilities instead of improving representative aspects of the school. The ``Cafeteria'' should be developed further and the a dialogue should be initiated with the APEE to improve the canteen. The high tuition quality in the higher years should be maintained and possibilities to hire more qualified staff for the lower years should be investigated. Efforts to raise awareness for environmental problems should be complemented with real action such as improving facilities for cyclists.


Leon Handreke, 29th April 2010


\end{document}

% LocalWords:  Tutition noticably pannels Handreke
