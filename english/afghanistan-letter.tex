\documentclass{letter}
\usepackage[top=1cm]{geometry}                % See geometry.pdf to learn the layout options. There are lots.
\geometry{a4paper}                   % ... or a4paper or a5paper or ... 
\usepackage{setspace}
%\geometry{landscape}                % Activate for for rotated page geometry
\usepackage{helvet}
\usepackage[T1]{fontenc}
\usepackage[utf8]{inputenc}

\parindent 1cm

\address{Leon Handreke \\
Sous-le-pont 1 \\
Brussels \\
Belgium}
\signature{Leon Handreke}
\begin{document}
\onehalfspacing
\fontfamily{phv}\selectfont

\begin{letter}{The Editor \\
1 Scott Place \\
Manchester M3 3GG \\
United Kingdom}

\opening{Dear Sir or Madam,}

I would like to comment on the article about education in Afghanistan in a recent issue of your newspaper, ``The Observer''. To me, the article seemed to suggest that while the situation in the north was good, in the south it was completely hopeless. Despite the fact that the situation in this part of the country is very bad and likely to even deteriorate, I would like to suggest why it may still be a worthwhile endeavor to continue efforts to improve education in the south, despite the seeming impossibility of success.

I think that, should the war in Afghanistan be over some day, an educated workforce would be a considerable advantage for the economy. An economic upturn could provide much needed stability in the times after the war and make Afghanistan a stabilising force in the middle east.

Another important reason for improving education is to support the emancipation of women in the traditional society in Afghanistan. Educating girls, however impossible it may seem at the present time, could be the key to a change of values and traditions in the Afghan society that seems to be needed to resolve the conflict effectively.

I also think that changing the attitudes of the young generation, much of the ethnic conflicts that form at least part of the base for this war could be resolved. All the soldiers in the world will not be able to bring change in society - the change has to happen from the inside.

I hope I was able to outline some of the reasons why I believe efforts to improve education in Afghanistan despite the seemingly hopeless situation described in your article.

\closing{Yours faithfully,}
\end{letter}

\end{document} 

