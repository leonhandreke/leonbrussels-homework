\documentclass[11pt]{article}
\usepackage{geometry}                % See geometry.pdf to learn the layout options. There are lots.
\geometry{letterpaper}                   % ... or a4paper or a5paper or ... 
\usepackage{setspace}
%\geometry{landscape}                % Activate for for rotated page geometry
\usepackage[parfill]{parskip}    % Activate to begin paragraphs with an empty line rather than an indent
\usepackage{graphicx}

\usepackage{helvet}
\usepackage{nopageno}

%\usepackage[german]{babel}
\usepackage[T1]{fontenc}
\usepackage[utf8]{inputenc}


\DeclareGraphicsRule{.tif}{png}{.png}{`convert #1 `dirname #1`/`basename #1 .tif`.png}

\title{Baccalaureate 2008 - Part II, Question 1}
\author{Leon Handreke}
\date{}                                           % Activate to display a given date or no date

\begin{document}

%double line spacing
\doublespacing

\maketitle
\fontfamily{phv}\selectfont

The newspaper article is about a new drug which can be used to modify the intensity of bad memories. Researchers in North America have discovered that the drug U0126, which was originally designed to treat heart disease, introduces amnesia and can therefore be used to weaken memories. Experiments with rats have shown that the drug can be used to effectively erase one memory while keeping others intact. However, the author also raises concerns about the use of the drug on humans, arguing that memories are an integral part of the personality and giving several examples of how unpleasant memories have benefited future generations.
\end{document} 

