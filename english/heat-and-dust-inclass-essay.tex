\documentclass[11pt]{article}
\usepackage{geometry}                % See geometry.pdf to learn the layout options. There are lots.
\geometry{letterpaper}                   % ... or a4paper or a5paper or ... 
\usepackage{setspace}
%\geometry{landscape}                % Activate for for rotated page geometry
\usepackage[parfill]{parskip}    % Activate to begin paragraphs with an empty line rather than an indent
\usepackage{graphicx}

\usepackage{helvet}
\usepackage{nopageno}

%\usepackage[german]{babel}
\usepackage[T1]{fontenc}
\usepackage[utf8]{inputenc}


\DeclareGraphicsRule{.tif}{png}{.png}{`convert #1 `dirname #1`/`basename #1 .tif`.png}

\title{Heat and Dust - In Class Essay}
\author{Leon Handreke}
\date{}                                           % Activate to display a given date or no date

\begin{document}

%double line spacing
\doublespacing

\maketitle
\fontfamily{phv}\selectfont

The themes of 'choice' and 'acceptance' play a major role in the novel ``Heat and Dust''. They are often used as counterparts to eachother in the two parallel storylines that have many things in common. This essay aims to show how each of the storylines deal with these themes.

The novel begins with the arrival in India of both the narrator and Olivia. While the narrator chooses to go to India, possibly in connection to the 'time out' that many Brits were taking in India at that time, Olivia is following her husband to India, accepting that she has to go. This is not saying that she does not want to go; As the narrator, she has a positive attitude towards India and is in no way opposed to living there.

Another example of Olivia's acceptance is the time she spends after her arrival, meeting with other British ladies and being at home most of the time. She accepts that this is the way it is at that time, although she does complain to her husband Douglas about her boredom. In contrast to this, the narrator can choose what she does. She walks around in the city and has and active lifestyle, choosing how she spends her time.

These two examples illustrate that Olivia's storyline is a lot more related to the theme of 'acceptance' and the narrator chooses a lot more things for herself. However, throughout the book there is an evolution in the themes used. Olivia starts to make more choices for herself, while the theme of acceptance is more and more introduced into the narrator's storyline.

An example for Olivia's evoltion from 'acceptance' to 'choice' are her increasing visits to the palace. Although Douglas and all the other British officers do not have a positive attitude towards the Nawab, she continues to visit his Palace. She actively chooses to do so and percieves the freedom to do so as relieving, making her more happy. At the same time in the book, the narrator meets Chid, whom she accepts living and even sleeping with her. She never chooses to keep him but just accepts that he is there.

The narrator also learns to accept some of the aspects of India, illustrated by her experience with the dying beggar woman. Olivia, on the other hand, gets even more involved with the theme of 'choice' by choosing to have an abortion, rather than accepting her child the way it is.

In conclusion, both storylines evolve in terms of the themes used. While Olivia's storyline evolves from acceptance to choice, the theme of acceptance is introduced into the narrator's storyline while still keeping choice the dominant theme. This 'emancipation' of Olivia could also be symbolic for the general emancipation of women. Towards the end of the book, the theme of choice gets increasingly dominant, making the evolution from 'acceptance' to 'choice' a central part of the book.

\end{document} 

