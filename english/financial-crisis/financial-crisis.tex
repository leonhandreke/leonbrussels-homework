\documentclass{beamer}

\usetheme{Hannover}
\usecolortheme{dove}

\usepackage{graphicx}
\usepackage{helvet}
\usepackage{url}

%\DeclareGraphicsRule{.tif}{png}{.png}{`convert #1 `dirname #1`/`basename #1 .tif`.png}

\title{The Financial Crisis}
\author{Leon Handreke}
\date{}

\begin{document}

\frame{\titlepage}

\frame{\tableofcontents}

\section{Causes}
\subsection{The Housing Bubble}

\frame{
  \frametitle{The Housing Bubble}
  \includegraphics[width=\textwidth]{house-price-increase.png}
}

\frame{
  \frametitle{The Housing Bubble}
  \begin{itemize}
  \item Price of typical American house increased rapidly from 1997 - 2006
  \item ``Refinancing'' of many homes
  \end{itemize}
}

\subsection{Sub-prime lending}
\frame{
  \frametitle{Sub-prime lending}
  \begin{itemize}
  \item ``Toxic'' loans
  \item Often financed by housing bubble buildup
  \item 10\% in 2004, 20\% of total loans in 2006
  \end{itemize}
}

\frame{
  \begin{quote}
    Fannie Mae, the nation's biggest underwriter of home mortgages, has been under increasing pressure from the Clinton Administration to expand mortgage loans among low and moderate income people and felt pressure from stock holders to maintain its phenomenal growth in profits. In moving, even tentatively, into this new area of lending, Fannie Mae is taking on significantly more risk, which may not pose any difficulties during flush economic times. But the government-subsidized corporation may run into trouble in an economic downturn, prompting a government rescue similar to that of the savings and loan industry in the 1980's.
  \end{quote}
  Steven A. Holmes, New York Times, September 30, 1999
}

\subsection{Non-regulation}

\frame{
  \frametitle{Non-regulation}
  \begin{itemize}
  \item Deregulation of financial markets throughout the past 30 years
  \item ``Futures''
  \item Easy credit conditions encourage high-risk operations
  \end{itemize}
}

\subsection{Human Factors}
\frame{
  \frametitle{Human Factors}
  \begin{itemize}
  \item Increasing complexity of the financial markets
  \item Huge bonuses
  \item Greed
  \item Loss of proportion
  \end{itemize}
}

\frame{
  \begin{center}
    \Huge Collapse
  \end{center}
}


\section{Effects}

\subsection{Credit Crunch}

\frame{
  \frametitle{Credit Crunch}
  \begin{itemize}
  \item No Equity ($\rightarrow$ Cash)
  \item No loans given
  \item ``Flow of money'' interrupted
  \item Effects felt on a microeconomic scale
    \begin{itemize}
    \item No loans for small companies
    \item Loss of employment
    \item Loss of housing
    \end{itemize}
  \item Countries collapsing
  \end{itemize}
}

\frame{
  \frametitle{Job losses}
  \includegraphics[width=\textwidth]{joblosses.png}
}

\subsection{Bailouts}

\frame{
  \frametitle{Bailouts}
  \begin{itemize}
  \item Government buys ``toxic'' assets from banks
  \item Equity is increased
  \item More loans given
  \end{itemize}
}


\frame{
  \begin{center}
    \huge Too Complex
  \end{center}
}

\section{Future Prevention}
\frame{
  \frametitle{Ideas for Future Prevention}
  \begin{itemize}
  \item Split up system-relevant institutions
  \item Require more transparency from banks
  \item Less sub-prime loans
  \item Limit responsibility and bonuses for individuals
  \end{itemize}
}

\section{My Crazy Ideas}

\frame{
  \begin{center}
    \Huge Disclaimer
  \end{center}
}

\frame{
  \frametitle{Public debt of the world}
  \includegraphics[width=\textwidth]{public-debt-world-map.png}
}

\frame{
  \frametitle{Increasing public and private debt}
  \begin{itemize}
  \item EU public debt: 61.5\% of GDP
  \item US public debt: 86.1\% of GDP
  \item Italy public debt: 105.8\% of GDP
  \item Japan public debt: 170\% of GDP
  \end{itemize}
}

\frame{
  \frametitle{Who ``owns'' public loans?}
  \includegraphics[width=\textwidth]{public-debt-ownership.jpg}
}

\frame{
  \frametitle{The Federal Reserve}
  \begin{itemize}
  \item US central bank
  \item ``As federal as Federal Express''
  \item Owned by banks and other shareholders
  \item 97\% of profits payed back to the government
  \end{itemize}
}

\frame{
  \frametitle{Debt}
  \begin{itemize}
  \item Do we own our own debt?
  \item Where is the deficit?
  \end{itemize}
}

\frame{
  \begin{center}
    \huge I don't know
  \end{center}
}

\section{Sources}
\frame{
  \frametitle{Sources}
  \begin{itemize}
  \item Wikipedia
  \item Department of Treasury, Treasury Bulletin, December 2009
  \item \url{http://www.nytimes.com/1999/09/30/business/fannie-mae-eases-credit-to-aid-mortgage-lending.html}
  \item \url{http://swampland.blogs.time.com/2009/02/06/how-bad-is-it/}
  \end{itemize}
}


\section{Questions}
\frame{
  \frametitle{Questions}
  \begin{itemize}
  \item What is meant by ``refinancing'' your home?
  \item What are sup-prime loans?
  \item What is meant by ``The Credit Crunch''?
  \end{itemize}
}

\end{document}
