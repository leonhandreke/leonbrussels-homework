\documentclass[10pt]{article}
\usepackage{setspace}
\usepackage[margin=2cm]{geometry}              % See geometry.pdf to learn the layout options. There are lots.
\geometry{a4paper}                   % ... or a4paper or a5paper or ... 
%\geometry{landscape}                % Activate for for rotated page geometry
\usepackage[parfill]{parskip}    % Activate to begin paragraphs with an empty line rather than an indent
\usepackage{graphicx}
\usepackage{helvet}
\usepackage{nopageno}
\usepackage{url}

\usepackage[german]{babel}
\usepackage[T1]{fontenc}
\usepackage[utf8]{inputenc}

\DeclareGraphicsRule{.tif}{png}{.png}{`convert #1 `dirname #1`/`basename #1 .tif`.png}

\title{Richard Dawkins}
\author{Leon Handreke}
\date{}                                           % Activate to display a given date or no date

\begin{document}

\maketitle

\fontfamily{phv}\selectfont

%\linespread{0.9}

\section{Biographie}

\begin{itemize}
\item Geboren am 26. M"arz 1941 in Nairobi, Kenia
\item Britische Staatsb"urgerschaft
\item Bis 1966: Studium der Zoologie in Oxford
\item Bis 1970: Assistenzprofessur an der Universit"at Berkeley, Kalifornien
\item Bis 2008: Professur an der Universit"at Oxford
\item 2006: Gr"undung der Richard Dawkins Foundation for Reason and Science
\end{itemize}


\section{Soziobiologie}

\begin{itemize}
\item Der K"orper als "Uberlebensmaschine, gesteuert durch die Gene
\item Das ``egoistische Gen'' als Menge aller Kopien in allen Organismen
\item Das einzelne Gen kann Kopien seiner selbst helfen
\item Bewusstsein ist eine Eigenschaft der ``modernen "Uberlebensmaschinen''
\end{itemize}

\section{Andere Theorien}
\begin{itemize}
\item ``In order not to believe in evolution you must either be ignorant, stupid or insane.''
\item Kulturelle Evolution: Mem als kulturelles "Aquivalent des Gens
\item Religion als Mem
\end{itemize}



\section{Quellen}
\begin{itemize}
\item \url{http://en.wikipedia.org/wiki/Richard_Dawkins}, Version vom 29.11.2009
\item \url{http://richarddawkins.net/}
\item ``Zug"ange zur Philosophie'', S. 134 ff.
\end{itemize}

\end{document}
