\documentclass[11pt]{article}
\usepackage{geometry}                % See geometry.pdf to learn the layout options. There are lots.
\geometry{a4paper}                   % ... or a4paper or a5paper or ...
%\geometry{landscape}                % Activate for for rotated page geometry
\usepackage[parfill]{parskip}    % Activate to begin paragraphs with an empty line rather than an indent
\usepackage{graphicx}
\usepackage{setspace}

\usepackage{helvet}
\usepackage{nopageno}
\usepackage{fullpage}

\usepackage[T1]{fontenc}
\usepackage[utf8]{inputenc}

\DeclareGraphicsRule{.tif}{png}{.png}{`convert #1 `dirname #1`/`basename #1 .tif`.png}

\title{History Homework on the Cuban Missile Crisis}
\author{Leon Handreke}
\date{}                                           % Activate to display a given date or no date

\begin{document}
\onehalfspacing
\maketitle
\fontfamily{phv}\selectfont

\section{Question 1}
The different sources give different reasons for why the Soviet Union might have chosen to position nuclear missiles in Cuba:

\begin{itemize}
\item In document A1 the Soviet Union's former foreign secretary names political reasons as the primary reasons for the nuclear presence in Cuba. He says that it was primarily a reaction to the USA's continued refusal to remove nuclear missiles from eastern Europe (``placement [...] was undertaken only after the United States [...] continually rejected proposals to remove [military installations] on foreign territory'').

\item In document A2, Fidel Castro cites the protection of Cuba as the primary reason for the nuclear missiles on Cuba (``make it clear to the United States that an invasion of Cuba would imply a war with the Soviet Union.''). I think that his answer comes without surprise - for him the missiles were a good protection against a possible invasion. However, I think that he is simply a straw man in the conflict between the two superpowers, repeating the official position of his ally instead of saying the truth.

\item In document A3, the former leader of the Soviet Union, Khrushchev, gives two reasons for the nuclear missiles: Protecting Cuba (``protect Cuba's existence as a Socialist country'') as well as exerting political pressure on the United States (``they would learn just what it feels like to have enemy missiles pointing at you'').

\item In document A4, David Detzer suggests that the nuclear warheads may have been a hoax (``It seems at least possible that the Russians were bluffing.'') and sees political pressure as the only reason for the missiles (``If the Kremlin's purpose was essentially political [...]'').

\item Document A5 suggest that the nuclear missiles on Cuba may have been there for strategic reasons. It compares the number of missiles the two superpowers were in possession of (``The United States [...] would have in the neighborhood of 1500 ballistic missiles... The total number of Soviet missiles [...] was about 125'') and points out the strategic importance of the missiles in Cuba (``By moving [...] missiles to Cuba [...] Russia was rapidly narrowing the gap'').
\end{itemize}


\section{Question 2}

The cartoon shows the Soviet leader Khrushchev preparing to pull missile-shaped teeth out of Fidel Castro's mouth. The cartoon refers to Khrushchev' decision to remove the nuclear missiles from Cuba during the Cuban Missile Crisis. The caption ``This hurts me more than it hurts you!'' refers to the fact that the removal of the missiles not only meant that Cuba was less protected (which did not please Fidel Castro) but also showed that Khrushchev was weak - giving in to the demands of the US hurt his political reputation significantly. The fact that the concession the US made to the Soviet Union to remove missiles from Turkey remained secret only made this effect of the Cuban Missile Crisis worse.
\end{document}
