\documentclass[11pt]{article}
\usepackage{geometry}                % See geometry.pdf to learn the layout options. There are lots.
\geometry{letterpaper}                   % ... or a4paper or a5paper or ... 
%\geometry{landscape}                % Activate for for rotated page geometry
\usepackage[parfill]{parskip}    % Activate to begin paragraphs with an empty line rather than an indent
\usepackage{graphicx}

\usepackage{helvet}
\usepackage{nopageno}

%\usepackage[german]{babel}
\usepackage[T1]{fontenc}
\usepackage[utf8]{inputenc}


\DeclareGraphicsRule{.tif}{png}{.png}{`convert #1 `dirname #1`/`basename #1 .tif`.png}

\title{History Homework on World War II}
\author{Leon Handreke}
\date{}                                           % Activate to display a given date or no date

\begin{document}
\maketitle
\fontfamily{phv}\selectfont

\section{Question 1}
\paragraph{}
The sources show many points of criticism that Russia had of the policy of the western democracies towards the facist agressors Italy and Germany. Source A shows the four western powers, Germany, Italy, Britain and France negotiating the Munich Agreement. Stalin stands in the door, wondering why there is no chair for him. It shows that Russia was angry at the western democracies for negotiating and even allying with fascist dictators but excluding the communist dictator, Stalin.

\paragraph{}
Source B shows the result of the Munich Agreement from Russias perspective. Because Hitler had been allowed to annex the Sudetenland, they felt that the western democracies had effectively allowed further expansion to the east. The cartoon shows Britain and France as cowards who want to direct Germany's destructive force anywhere but their own countries. This was seen as a very egoistic behaviour towards Russia and the eastern european countries in general.

\paragraph{}
Source C shows the result of the Munich Agreement that Russia was probably least amused about: An alliance of Germany, Italy, Britain and France meant that Russia would be alone in case of a war. While the Munich Agreement was not exactly an alliance of the four countries, it was not an advantage to Russia's position in the following war either.

\paragraph{}
All three sources show that Russia was very unsatisfied with the way things were going between the western powers and the Munich Agreement especially. The distrust built up during this period also contributed to the isolationist attitude of Russia after the war when Russia estabilshed a communist empire. Their complete rejection of capitalism during that time may have been a kind of revenge for the rejection of communism of the western democracies at the cost of allying with fascists in the period leading up to the second world war.

\section{Question 2}
\paragraph{}
While Britain was well aware of the situation in Czechoslovakia, source D shows that many did not consider it as a very important issue for Britain, mainly due to the physical distance. Especially source E shows that Churchill already saw further expansion of the German Empire coming but may not have considered it as an immediate threat to Britain. Obviously both Britain and France dramatically underestimated Germany and Hitlers desire of expansion.

\section{Question 3}
\paragraph{}
The experiences made in the policy of Appeasement towards Hitler and Mussolini dominated the Western attitude in the cold war. They had hoped that by giving Hitler what he wanted, they could satisfy him and avoid a war. However, this strategy tourned out to have disasterous effects; The lesson learnt was to never make any concessions to dictators. This strategy dominated the cold war and the policies made against the communist empire.

\end{document}  
