\documentclass[11pt]{article}
\usepackage{geometry}                % See geometry.pdf to learn the layout options. There are lots.
\geometry{a4paper}                   % ... or a4paper or a5paper or ...
%\geometry{landscape}                % Activate for for rotated page geometry
\usepackage[parfill]{parskip}    % Activate to begin paragraphs with an empty line rather than an indent
\usepackage{graphicx}
\usepackage{setspace}

\usepackage{helvet}
\usepackage{nopageno}
\usepackage{fullpage}

\usepackage[T1]{fontenc}
\usepackage[utf8]{inputenc}

\topmargin 1cm

\DeclareGraphicsRule{.tif}{png}{.png}{`convert #1 `dirname #1`/`basename #1 .tif`.png}

\title{History Homework on the Korean War}
\author{Leon Handreke}
\date{}                                           % Activate to display a given date or no date

\begin{document}
\onehalfspacing
\maketitle
\fontfamily{phv}\selectfont

\section{Question 1}
The phrase ``Died of lack of exercise facing wanton aggression'' on the headstone is a reference to the League of Nation's weakness prior to World War II. The aggressions include the growing power of Germany and it's breach of the the Treaty of Versailles. The remilitarization of the Rheinland by Germany and the ``Anschluss'', in which Austria and Germany joined to create one large state are examples of events in which the League of Nations could have intervened but did not. Instead, a policy of appeasement was used, which ultimately failed to prevent World War II. As a consequence, the League of Nations became insignificant and was finally succeeded by the United Nations. The author calls this behavior a ``lack of exercise'' and cites this as the reason for it's death.

\section*{Question 2}
The man rushing towards the tomb is Harry S. Truman, who at that time was the President of the United States. He is accompanied by a woman representing the United Nations organization.

\section*{Question 3}
The cartoon shows the United Nations behaving in exactly the opposite way as it's predecessor, the League of Nations. Lead by Truman, the woman representing the United Nations is carrying a machine gun, preparing to fight in the Korean War. She is doing the opposite to the United Nations: Instead of letting North Korea expand it's territory (appeasement), the United States, in the name of the United Nations, intervened and defended South Korea, temporarily even conquering it almost completely.

The position the cartoon is taking is probably somewhere in between the two extremes: The United Nations organization is shown as extremely aggressive, the League of Nations as too weak. The message of the cartoon may be that a position in between these two extremes is better: Intervene and defend without going into the offensive like the United Nations (lead by the United States) did in the Korean War.

\section*{Question 4}
The second cartoon illustrates the position the UN is taking in the Korean War. The man walking across the street, signaling the Chinese tank to stop, represents the American army that is crossing the 38th parallel, the former boundary between South and North Korea. The kid he is taking across the street with him represents the United Nations and the resolution that was passed by it, allowing the American army to cross this boundary in the name of the United Nations forces (lead by American general McArthur). The man signals the Chinese tank to stop because they are also bound by the United Nations resolution and may not directly attack the UN forces.

\section*{Question 5}
The incident described in Document 3 happened toward the end of the Korean War when the UN forces attempted one final offensive, eventually halting at the 38th parallel. This can be derived from the date of the incident (16th of January 1953). Also, the location of the city mentioned, Kumhwa, is close to the current border, indicating that the front at that time was around the 38th parallel. This was only the case towards the end of the war.

\section*{Question 6}
The article says that the Chinese forces, after defeating their enemies, ``started looting the dead and executing the wounded''. The Geneva Conventions, initially conceived in 1864, require wounded soldiers to be treated humanely and explicitly state that they may not be killed. They also require the fighting parties to record the identities of the dead soldiers and transmit them to the enemy. The Chinese soldiers did not record the identity of the soldier posing as dead described in the article, indicating that this convention was ignored systematically. However, China only signed the Geneva Convention in 1956, after the Korean War had ended and was therefore not bound by it at the time of the described event.

\end{document}
