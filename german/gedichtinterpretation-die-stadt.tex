\documentclass[11pt]{article}
\usepackage{geometry}                % See geometry.pdf to learn the layout options. There are lots.
\geometry{letterpaper}                   % ... or a4paper or a5paper or ... 
%\geometry{landscape}                % Activate for for rotated page geometry
\usepackage[parfill]{parskip}    % Activate to begin paragraphs with an empty line rather than an indent
\usepackage{graphicx}
\usepackage{setspace}

\usepackage{helvet}
\usepackage{nopageno}

\usepackage[german]{babel}
\usepackage[T1]{fontenc}
\usepackage[utf8]{inputenc}


\DeclareGraphicsRule{.tif}{png}{.png}{`convert #1 `dirname #1`/`basename #1 .tif`.png}

\title{Gedichtinterpretation \dq Die Stadt\dq }
\author{Leon Handreke}
\date{}                                           % Activate to display a given date or no date

\begin{document}

\maketitle

\fontfamily{phv}\selectfont
\doublespacing

Das Gedicht \dq Die Stadt\dq\  von Georg Heym wurde im Jahre 1910 zur Zeit der industriellen Revolution verfasst. Es thematisiert  dem Titel entsprechend eine Stadt und greift damit ein zentrales Motiv der Literaturepoche des Expressionismus auf. Dieser Aufsatz soll zeigen, inwiefern \dq Die Stadt\dq\  ein typisches Gedicht des Expressionismus ist.

Das Gedicht ist als Sonett verfasst und thematisiert hauptsächlich das Motiv der Stadt. Die beiden ersten Strophen beschreiben die Stadt also großes, jedoch ruhiges Konstrukt (\dq Sehr weit ist diese Nacht... tausend Fenster... blinzeln mit den Lidern, rot und klein.\dq ). Die Fenster werden personifiziert, die Menschen jedoch zu einer Masse abstrahiert; Ein weiteres typisches Merkmal des Expressionismus.

Nach der zweiten Strophe lässt sich eine Zäsur feststellen: Es wird jetzt nicht mehr direkt die Stadt beschrieben; Stattdessen folgen gegensätzliche Nomen aufeinander (\dq Gebären, Tod, gewirktes Einerlei...\dq ). Auch die Atmosphäre verändert sich abrupt: Sie wird wesentlich negativer und düsterer, Ausdrücke wie \dq Sterbeschrei\dq , \dq blinder Wechsel\dq\  und \dq drohn im Weiten\dq\  unterstützen die bedrohliche Atmosphäre. In der letzten Strophe wird ein Brand beschrieben. Auch dieser wird als bedrohlich beschrieben. Das Motiv der Naturkatastrophe ist typisch für den Expressionismus und wird in verschiedenen Formen in vielen Gedichten aus dieser Zeit verwendet.

Des weiteren werden in dem Gedicht überwiegend Substantive verwendet - Adjektive und Verben spielen eine untergeordnete Rolle. Man findet viele unvollständige Sätze, überwiegend im zweiten, düstereren Teil. Trotzdem ist das Gedicht sehr strukturiert - umarmende Reime werden durchgängig verwendet und das Schema des Sonetts wird streng eingehalten.

Abschließend lässt sich feststellen, dass \dq Die Stadt\dq\  tatsächlich ein typisches Gedicht des Expressionismus ist. Seine Vermenschlichung der Stadt, Abstraktion des Menschen, Verwendung vieler Substantive sowie das Motiv der Naturkatastrophe machen es zu einem typischen Beispiel für diese Literaturepoche.


\end{document}  
