\documentclass[11pt]{article}
\usepackage{setspace}
\usepackage{geometry}                % See geometry.pdf to learn the layout options. There are lots.
\geometry{letterpaper}                   % ... or a4paper or a5paper or ... 
%\geometry{landscape}                % Activate for for rotated page geometry
\usepackage[parfill]{parskip}    % Activate to begin paragraphs with an empty line rather than an indent
\usepackage{graphicx}

\usepackage{helvet}
\usepackage{nopageno}

\usepackage[german]{babel}
\usepackage[T1]{fontenc}
\usepackage[utf8]{inputenc}


\DeclareGraphicsRule{.tif}{png}{.png}{`convert #1 `dirname #1`/`basename #1 .tif`.png}

\title{Romantik - Zwischen Traum und Wirklichkeit}
\author{Leon Handreke}
\date{}                                           % Activate to display a given date or no date

\begin{document}

\maketitle

\fontfamily{phv}\selectfont

\doublespacing

\section{Novalis - Wenn nicht mehr Zahlen und Figuren}

Das Gedicht "Wenn nicht mehr Zahlen und Figuren" kritisiert die Naturwissenschaften sowie diese die sie betreiben. In den Versen 1 bis 4 wird die These aufgestellt, dass Künstler die Welt besser erklären könnten als die "Tiefgelehrten" Naturwissenschaftler. Das Gedicht schlägt in den letzten Versen Methoden einer alternativen Wahrheitsfindung vor, in denen sich stark Motive der Romantik erkennen lassen. So wird zum Beispiel vorgeschlagen, "Licht und Schatten", die Vernunft am Tag und die Freiheit und Mystik in der Nacht, zu verschmelzen, um zu einer wahren Erkenntnis zu kommen. Das Gedicht stellt die These auf, dass "Märchen und Gedichte" die wahre Geschichte der Welt erkennen lassen.

\section{Zwischen Traum und Wirklichkeit}

Die Romantische Kunst ist als Gegenreaktion auf die Epoche der Aufklärung entstanden, die stark von der Vernunft und einem festen Gesellschaftsgefüge geprägt war. Das damals noch existierende "Heilige Römische Reich deutscher Nation" was von Fürsten und anderen absolutistischen Herrschaftsträgern regiert, die das Volk oft unterdrückten. Die Französische Revolution rückte das Thema der Bürgerrechte und der Freiheit das erste mal seit langem in die öffentliche Wahrnehmung. Auch die Intellektuellenbewegung der Romantik stützte sich Teilweise auf die Prinzipien "Freiheit, Gleichheit, Brüderlichkeit". Diese Revolution in den Köpfen der Menschen kam auch in den Werken der Romantik zum Ausdruck, die oft sehr emotional und von Motiven der Freiheit dominiert waren. Diese Bewegung wurde auch durch den nachfolgenden Krieg gegen Napoleon weiter voran getrieben, da damit ein Feindbild geschaffen werden konnte. Diese und andere Faktoren trugen dazu bei, dass die Kulturepoche der Romantik bis in das 19. Jahrhundert hinein Kritik am bestehenden System. Als die Bewegung jedoch in den frühen 20er Jahren des 19. Jahrhunderts zu groß und damit gefährlich für die Fürsten und das angestammte politische System wurde, begannen die Herrschaftsträger, Demokraten und andere politische Aktivisten gezielt zu verfolgen und setzten der Romantik damit weitgehend ein Ende.

\end{document}  
