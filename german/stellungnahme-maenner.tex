\documentclass[11pt]{article}
\usepackage{setspace}
\usepackage{geometry}                % See geometry.pdf to learn the layout options. There are lots.
\geometry{a4paper}                   % ... or a4paper or a5paper or ... 
%\geometry{landscape}                % Activate for for rotated page geometry
\usepackage[parfill]{parskip}    % Activate to begin paragraphs with an empty line rather than an indent
\usepackage{graphicx}

\usepackage{helvet}
\usepackage{nopageno}

\usepackage[german]{babel}
\usepackage[T1]{fontenc}
\usepackage[utf8]{inputenc}


\DeclareGraphicsRule{.tif}{png}{.png}{`convert #1 `dirname #1`/`basename #1 .tif`.png}

\title{Stellungnahme - ``Männer - Das bequeme Geschlecht''}
\author{Leon Handreke}
\date{}                                           % Activate to display a given date or no date

\begin{document}

\maketitle

\fontfamily{phv}\selectfont

\doublespacing

In dem vorliegenden Text ``Männer - das faule Geschlecht'' von Claudia Kester wird die Hauptthese vertreten, dass Männer entgegen des gängigen Klischees ``fauler'' sind als Frauen. Der Autor unterstützt diese These mit vielen Autoritätsargumenten gestützt auf Statistiken des Statistischen Bundesamtes. Es wird von diesen Statistiken belegt, dass seit 1990 kein wesentliche Veränderung in der Menge an unbezalter Arbeit die durch Männer verrichtet wird stattgefunden hat. Des weiteren wird belegt, dass Frauen im durchschnitt pro Woche etwa eine Stunde länger Arbeiten als Männer.

Das Gegenargument, dass die Verweigerung der unbezahlten Arbeit durch Männer oft zu einem Imageschaden führt und deshalb mehr als früher verrichtet wird entkräftet der Autor mit der Feststellung, dass Männer oft die angenehmen und vergnüglichen Arbeiten machen und oft ablehnen selbst Verantwortung zu übernehmen. Des weiteren wird behauptet, Männer würden sich oft Ausreden bedienen, um nicht im Haushalt mitarbeiten zu müssen. Das Argument, dass Männer deutlich mehr Zeit mit Heimwerken zubringen wird dadurch entkräftet, dass nach dem Statistischen Bundesamt nur 12\% aller unbezahlter Arbeit in diesen Sektor fällt und damit im Ganzen insignifikant ist. Das Argument, dass Männer mehr Zeit mit ihren Kindern verbringen als früher wird mit der Behauptung, dass Männer oft nur die schönen Seiten der Kinderbetreuudg übernehemen wollen erwidert.

Des weiteren stellt der Autor die These auf, dass unbezahlt Arbeit in der Gesellschaft zu wenig anerkannt wäre und dass ein Umdenken sowohl bei den Männern als auch in der Gesellschaft allgemein stattfinden müsse. Als Beispiel hierfür nennt er die Notwendigkeit, schon in der Schule die Bildung von Klischees zu unterbinden. Zur Untermauerung der Nebenthese, dass eine Änderung der Gesetzeslage keine Änderung im Verhalten der Männer hervorrufe, nennt er das Beispiel des nach seiner Meinung ineffektiven Gleichstellungsdurchstezungsgesetzes.

Das Autor schließt mit der Forderung ab, dass auch für unbezahlte Arbeiten der Grundsatz ``Gleicher Lohn für gleiche Arbeit'' gelten sollte.

Ich stimme mit der These des Autors, dass unbezahlte Arbeit in der Gesellschaft einen höheren Stellenwert haben sollte überein. Eine solche Entwicklung hätte möglicherweise auch eine Auswirkung auf die Gesellschaft als Ganzes - mehr ehrenamtliche Arbeiten außerhalb des eigenen Haushaltes oder Gartens würden durch Männer und Frauen verrichtet werden, da sie auch im sozialen Umfeld anerkannt sind.

Skeptisch bin ich jedoch gegenüber der These, dass Männer das ``faulere'' Geschlecht sind. Meiner Meinung nach hängen viele der vom Autor beschrieben Sachverhalten mit der historisch bedingten Arbeitsteilung zusammen, nach der Frauen die ``einfacheren'' Arbeiten übernehmen. Erst durch komplette Gleichstellung auch in der Arbeitswelt kann eine Situation geschaffen werden, in der die Leistungen von Männern und Frauen vergleichbar sind. Allen Männern in Folge von nicht gelösten gesellschaftlichen Problemen das Attribut ``faul'' zu geben ist meiner Meinung nach zu unrecht diskriminierend.

Des weiteren ist der direkte Vergleich zwischen der Arbeitszeit von Männeren und Frauen, den der Autor anstellt meiner Meinung nach nicht ohne weiteres möglich. Während viele Frauen, die hauptsächlich unbezahlt arbeiten, sich ihre Arbeitszeiten aussuchen können und oft in einer angenehmeren Umgebung arbeiten, arbeiten Männer oft in einer wesentlich unflexibleren und unangenehmeren Umgebung. Dies macht die Belastung oft wesentlich größer, wodurch es unmöglich wird, die Arbeitszeit in ein direktes Verhältnis zu der tatsächlichen Leistung zu stellen.

Der Autor behauptet weiterhin, dass Männer oft Ausreden verwenden, um einen Imageschaden wegen verweigerter Arbeit im Haushalt zu vermeiden, kann diese Behauptung jedoch nicht belegen. Sie wirkt deshalb unglaubwürdig. Außerdem wird behauptet, dass Männer im Regelfall die angenehmen unbezahlten Arbeiten verrichten. Auch dies wird nicht belegt. Des weiteren wird in vielen der genannten Statistiken über die Menge geleisteter unbezahlter Arbeit statt der Qualität die Quantität betont - möglicherweise haben sich statt der Arbeitszeit seit 1990 vor allem die Aufgabenbereiche im Haushalt verschoben - dies geht aus der am Anfang des Textes herangezogenen Statistik nicht hervor. Auch das Gleichsetzungsdurchstellungsgesetz wird mit der unbelegten Behauptung, dass es wenig in die häuslichen Verhältnisse hinein strahlt, ohne weitere Hinterfragung oder Beweise, für ineffektiv erklärt.

Abschließend lässt sich sagen, dass der Autor des Textes zwar ein meiner Meinung nach valides Anliegen hat, durch seine reißerische Überschrift jedoch eine schwache Argumentation gegen alle Männer führen muss. Es werden viele grundlose Behauptungen aufgestellt und die Statistiken teilweise extrem einseitig interpretiert. Gesetzliche Regelungen werden kurzerhand für sinnlos erklärt, was den Autor arrogant erscheinen lässt. Die abschließende Äußerung, dass Männer das ``bequeme Geschlecht'' seien festigt bei mir den arroganten und selbstgerechten Eindruck des weiblichen Autors.

\end{document}
