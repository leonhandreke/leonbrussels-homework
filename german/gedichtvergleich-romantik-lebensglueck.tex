\documentclass[11pt]{article}
\usepackage{geometry}                % See geometry.pdf to learn the layout options. There are lots.
\geometry{letterpaper}                   % ... or a4paper or a5paper or ... 
%\geometry{landscape}                % Activate for for rotated page geometry
\usepackage[parfill]{parskip}    % Activate to begin paragraphs with an empty line rather than an indent
\usepackage{graphicx}
\usepackage{setspace}

\usepackage{helvet}
\usepackage{nopageno}

\usepackage[ngerman]{babel}
\usepackage[T1]{fontenc}
\usepackage[utf8]{inputenc}


\DeclareGraphicsRule{.tif}{png}{.png}{`convert #1 `dirname #1`/`basename #1 .tif`.png}

\title{Gedichtvergleich "Die zwei Gesellen" - "Rückschau"}
\author{Leon Handreke}
\date{}                                           % Activate to display a given date or no date

\begin{document}

\doublespacing

\maketitle

\fontfamily{phv}\selectfont


\paragraph{}
Die zwei Vorliegenden Gedichte, "Die zwei Gesellen"\ von Joseph von Eichendorff und "Rückschau"\ von Heinrich Heine stammen beide aus der Epoche der Romantik. Beide Gedichte erzählen Lebensgeschichten verschiedener Personen, die jedoch alle gleich mit dem Tod enden. Dieser Aufsatz soll die beiden Gedichte bezüglich ihrer Auffassung des Lebens und des Sterbens vergleichen.

\paragraph{}
Das Gedicht "Die zwei Gesellen"\ von Joseph von Eichendorff handelt von den verschieden Lebenswegen zweier Gesellen, die am Anfang des Gedichts beide das Elternhaus verlassen. Ihre Ausgangssituation wird in den ersten beiden Strophen verdeutlicht. Beide wollen "was Rechts vollbringen"\ und haben hohe Ziele, was ihren Optimismus reflektiert. Auch die Metapher des Frühlings (Z. 5: "Des vollen Frühlings hinaus.") illustriert die positive Gesamteinstellung beider Personen zum Leben.

Die dritte Strophe erzählt die Lebensgeschichte des ersten Gesellen, der sich sehr schnell niederlässt und ein traditionelles, geregeltes Leben führt. Er verkörpert in diesem Gedicht den Spießer, der rein nach der Vernunft handelt. Dieses Verhalten wurde von vielen Autoren der Romantik verurteilt und kritisiert. Trotzdem in diesem Gedicht keine direkte Kritik des Vernünftigen stattfindet, wird sein Leben als eher langweilig beschrieben (Z. 14/15: "Und sah aus heimlichen Stübchen / Behaglich ins Felde hinaus.").

Die folgenden zwei Strophen erzählen von der gescheiterten Seefahrerkarriere des zweiten Gesellen. Dieser Weg wird als der Verlockendere beschrieben (Z. 18: "Verlockend Sirenen und zogen / Ihn in der buhlenden Wogen"). Im Dunklen bleibt die Beschreibung der Erlebnisse des zweiten Gesellen, es wird jedoch am Ende der vierten Strophe angedeutet, dass er Abenteuer erlebt (Z. 20: "Farbig klingenden Schlund"). Die fünfte Strophe beschreibt sein Leben nach seinem "Auftauchen aus dem Schlund"\ (Z. 21). Es wird als trist, kalt und still beschrieben.

Die letzte Strophe kann als Reflexion der Handlung durch das lyrische Ich gesehen werden. Es ist traurig, wenn es neue Gesellen sieht die ähnlich wie die Beschriebenen in der Welt ihr Glück suchen, da es ihr Schicksal schon erahnen kann. (Z. 28: "Und seh ich so kecke Gesellen, /  Die Tränen im Auge mir schwellen."). Im letzten Vers äußert das lyrische Ich den Wunsch nach der Führung durch Gott. Über die Richtung dieser Führung kann man nur spekulieren, vermutlich wird diese jedoch einen Mittelweg zwischen den zwei dargestellten Extremen in der Lebensweise darstellen. Außerdem wird der Wunsch nach dem Tod geäußert, der als Erlösung gesehen wird (Z. 30: "Ach Gott, führ uns liebreich zu Dir!").

Das Gedicht verwendet als Reimform eine auf die fünfversigen Strophen angepasste Version des Kreuzreims. Außerdem werden viele Enjambements verwendet. Zusammen mit dem parataktischen Stil des Gedichts ergibt sich ein sehr strukturierter Gesamteindruck. Der Jambus wird verwendet, um dem eigentlich den Tod thematisierenden Gedicht einen positiven Rhythmus zu geben. 

Auch das Gedicht "Rückschau"\ von Heinrich Heine thematisiert das Leben und den abschließenden Tod. In den ersten Strophen wird das Leben der lyrischen Ichs in einer Traumwelt beschrieben. So werden die Erlebnisse und Errungenschaften aufgelistet (Z. 1 - 19), die jedoch nach einer Zuspitzung der Erzählung ins unglaubwürdige (Z. 18/19: "Mir flogen gebratne Tauben ins Maul / Und Englein kamen ...") als Traum  offenbart werden. Ähnlich wie bei "Die zwei Gesellen"\ nach der vierten Strophen findet ein Umschwung ins Negative und Traurige statt. Das lyrische Ich sagt, dass es sich jede Träumerei mit negativen Konsequenzen in der Realität hat erkaufen müssen (Z. 25/26: "... jeden Genuss / Hab ich erkauft durch herben Verdruss."). Des weiteren Spielt der Autor mit Elementen der Ironie (Z. 35/36: "Dort oben, ihr christlichen Brüder/ Ja, das versteht sich, dort sehen wir uns wieder.")  und deutet sogar seinen Selbstmord an. Diese beiden Aspekte lassen eine spöttische Haltung gegenüber der Religion erkennen, im krassen Gegensatz zu der ergebungsvollen Haltung gegenüber Gott des lyrischen Ichs in "Die zwei Gesellen". Auch die formalen Stilmerkmale unterscheiden sich bei den beiden Gedichten erheblich; "Rückschau"\ verwendet viele Paarreime, jedoch wenige Enjambements. Das Metrum ist jedoch bei beiden Gedichten gleich.

Abschließend lässt sich feststellen, dass "Die zwei Gesellen"\ mit seinen Religiösen Bezügen und starken Ideologie auch im Bezug auf das Thema Lebensstil und Tod das typischere Romantikgedicht ist. "Rückschau"\ grenzt sich mit seiner Religionskritik und seinem klaren Realismus klar von den träumerischen Tendenzen der Romantik und setzt sich demnach auch deutlich kritischer und extremer mit den Themen Leben und Tod auseinander.

\end{document}  
