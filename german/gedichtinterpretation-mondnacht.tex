\documentclass[11pt]{article}
\usepackage{geometry}                % See geometry.pdf to learn the layout options. There are lots.
\geometry{letterpaper}                   % ... or a4paper or a5paper or ... 
%\geometry{landscape}                % Activate for for rotated page geometry
%\usepackage[parfill]{parskip}    % Activate to begin paragraphs with an empty line rather than an indent
\usepackage{graphicx}
\usepackage{amssymb}
\usepackage{epstopdf}
\usepackage{helvet}


\DeclareGraphicsRule{.tif}{png}{.png}{`convert #1 `dirname #1`/`basename #1 .tif`.png}

\title{Gedichtinterpretation "Mondnacht"}
\author{Leon Handreke}
%\date{}                                           % Activate to display a given date or no date

\begin{document}

\maketitle

\fontfamily{phv}\selectfont

%\section{}
%\subsection{}

Das vorliegende Gedicht "Mondnacht" von Joseph von Eichendorff wurde im Jahre 1830 verfasst und f獲lt damit in den sp閣en Teil der Literaturepoche der Romantik. Trotz seiner relativen Kompaktheit lassen sich in dem Text viele Stilmerkmale ausmachen, die den Text eindeutig der Romantik zuordnen. Diese Gedichtinterpretation soll diese Stilmerkmale des Textes herausarbeiten und zeigen, inwiefern diese typisch fuer dessen Literaturepoche sind.

Offensichtlich stark von der Literaturepoche gepraegt ist der Inhalt des Gedichts. Hauptsaechlich in der ersten und zweiten Strophe wird ein intensives Naturerlebnis des lyrischen Ichs beschrieben. Besonders hervorgehoben wird die Stille und Ruhe der Natur. Dies geht sogar so weit, dass Umschreibungen fuer Woerter verwendet werden, die zu dynamisch wirken koennten ("Die Luft ging durch die Felder", "Luft" statt "Wind"). Auch andere Faktoren der Beschreibung der Natur, wie z.B die neue Wortschoepfung "Bluetenschimmer" deuten auf eine verherrlichende Beschreibung der Natur hin.

Ein weiteres subtiles Stilmittel ist die Verwendung des Jambus als Rythmus des Gedichts. Er verleiht den Beschreibungen trotz der Stille und der naechtlichen Szene eine positive Atmosphaere und raeumt jeden Gedanken an Trauer aus.

Ein interessantes Detail ist die Verwendung des Konjunktiv in der ersten und letzten Strophe. Demnach liefert nur die zweite Strophe eine faktische Beschreibung der Szene. Alle anderen Bescreibungen sind demnach subjektive Eindruecke des Autors ("Es war als haett der Himmel / Die Erde still gekuesst"). Die Verwendung des Konjunktiv traegt zu der emotionalen Atmosphaere des Gedichts bei. Auch dies ist ein Kennzeichen der Romantik, in der Gefuehle eine grosse Rolle spielten.

Ein weiteres wichtiges Element der Romantik, der sich auch in dem vorliegenden Gedicht wiederfindet ist der religioese Bezug. So lassen sich in der ersten Strophe Refenzen auf die im Christentum wichtigen Elemente des Himmels und der Erde finden. Auch der Bezug auf die Aehren in der zweiten Strophe kann als Referenz auf das im Christemtum wichtige Brot gedeutet werden.

Eine interessante Parallele zur Malerei in der Kulturepoche der Romantik ist die indirekte Beleuchtung der bschriebenen Szene. Das Wort "Bluetenschummer" deutet genau so wie der Titel des Gedichts, "Mondnacht" auf eine diffuse Lichtquelle hin. Solch eine indirekte Beleuchtung laesst sich auch in vielen Gemaelden der Romantik beobachten. Auch dort sind oft der Morgen oder die Abenddaemmerung ein Motiv. Des weiteren bleibt das lyrische Ich genau wie die auf den Gemaelden der Romantik oft abegebildeten Personen komplett anonym.

Schlussfolgernd laesst sich feststellen, dass "Mondnacht" von Joseph von Eichendorff ein typisches Romantikgedicht ist. Die vielen typischen Motive und Stilmittel, wie der Bezug auf die Natur und Religion, der hohe Grad an Emotionalitaet sowie die vielen Parallelen zur Malerei der Romantik erlauben das vorliegende Gedicht eindeutig der Romantik zuordnen zu koennen.




\end{document}  
